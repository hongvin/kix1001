\documentclass[12pt,fleqn]{beamer}
\usepackage{xcolor}
\usetheme{AnnArbor}
\usecolortheme{beaver}
\usepackage{setspace}
\usepackage{cancel}
\usepackage{tcolorbox}
\addtobeamertemplate{frametitle}{}{\vspace{-0.4em}} 
\newcommand{\question}[1]{\textcolor{blue}{#1}}
\setbeamertemplate{itemize items}[ball]
\makeatother
\title{KIX1001 Tutorial 1}
\author[Hong Vin]{}
\date{October 22, 2021}
\begin{document}

%%% Question 1 %%%
\begin{frame}[t]
\frametitle{Question 1}

\question{Find the limit of $\displaystyle \lim _{x\rightarrow 5}\frac{x^{2} -25}{x^{2} +x-30}$.}

\[
\begin{array}{@{}r @{{}={}} l@{}} 
\lim _{x\rightarrow 5}\frac{x^{2} -25}{x^{2} +x-30} & \lim _{x\rightarrow 5}\frac{\cancel{(x-5)} (x+5)}{\cancel{(x-5)} (x+6)}\pause\\[8pt]
 & \lim _{x\rightarrow 5}\frac{x+5}{x+6}\pause\\[8pt]
 & \frac{5+5}{5+6}\pause\\[8pt]
 & \mathbf{\frac{10}{11}}
\end{array}
\]
\end{frame}

%%% Question 2 %%%
\begin{frame}[t]
\frametitle{Question 2}

\question{Find the limit of $\displaystyle \lim _{x\rightarrow 9} \ \frac{\sqrt{x} -3\ }{x-9}$.}

\[
\begin{array}{@{}r @{{}={}} l@{}} 
\lim _{x\rightarrow 9}\frac{\sqrt{x} -3}{x-9} & \lim _{x\rightarrow 9}\frac{\sqrt{x} -3}{x-9} \times \frac{\sqrt{x} +3}{\sqrt{x} +3}\pause\\[8pt]
 & \lim _{x\rightarrow 9}\frac{\cancel{x-9}}{\cancel{(x-9)} (\sqrt{x} +3)}\pause\\[8pt]
 & \lim _{x\rightarrow 9}\frac{1}{(\sqrt{x} +3)}\pause\\[8pt]
 & \frac{1}{\sqrt{9} +3}\pause\\[8pt]
 & \mathbf{\frac{1}{6}}
\end{array}
\]
\end{frame}

%%% Question 3 %%%
\begin{frame}[t]
\frametitle{Question 3}

\question{Find the limit of $\displaystyle \lim _{x\rightarrow 0}\frac{\sqrt{2+x} -\sqrt{2}}{x}$.}

\[
\begin{array}{@{}r @{{}={}} l@{}} 
\lim _{x\rightarrow 0}\frac{\sqrt{2+x} -\sqrt{2}}{x} & \lim _{x\rightarrow 0}\frac{\sqrt{2+x} -\sqrt{2}}{x} \times \frac{\sqrt{2+x} +\sqrt{2}}{\sqrt{2+x} +\sqrt{2}}\pause\\[8pt]
 & \lim _{x\rightarrow 0}\frac{(\sqrt{2+x} )^{2} -(\sqrt{2} )^{2}}{x(\sqrt{2+x} +\sqrt{2} )}\pause\\[8pt]
 & \lim _{x\rightarrow 0}\frac{2+x-2}{x(\sqrt{2+x} +\sqrt{2} )}\pause\\[8pt]
 & \lim _{x\rightarrow 0}\frac{\cancel{x}}{\cancel{x} (\sqrt{2+x} +\sqrt{2} )}\pause\\[8pt]
 & \frac{1}{\sqrt{2} +\sqrt{2}}\pause\\[8pt]
 & \frac{1}{2\sqrt{2}} \times \frac{\sqrt{2}}{\sqrt{2}}\pause\\[8pt]
 & \mathbf{\frac{\sqrt{2}}{4}}
\end{array}
\]
\end{frame}

%%% Question 4 %%%
\begin{frame}[t]
\frametitle{Question 4}

\question{Find the limit of $\displaystyle \lim _{\theta \rightarrow \frac{\pi }{2}}\frac{\tan \theta }{\sec \theta }$.}

\[
\begin{array}{@{}r @{{}={}} l@{}} 
\lim _{\theta \rightarrow \frac{\pi }{2}}\frac{\tan \theta }{\sec \theta } & \lim _{\theta \rightarrow \frac{\pi }{2}}\frac{(\frac{\sin \theta }{\cancel{\cos \theta }} )}{(\frac{1}{\cancel{\cos \theta }} )}\pause\\[8pt]
 & \lim _{\theta \rightarrow \frac{\pi }{2}}\sin \theta\pause\\[8pt]
 & \sin\frac{\pi }{2}\pause\\[8pt]
 & \mathbf{1}
\end{array}
\]
\end{frame}

%%% Question 5 %%%
\begin{frame}[t]
\frametitle{Question 5}

\question{Find the limit of $\displaystyle \lim _{\theta \rightarrow 0}\frac{\cos \theta -1}{\sin \theta }$.}

\begin{columns}
\column{0.65\linewidth}
\begin{align*}
\lim _{\theta \rightarrow 0}\frac{\cos \theta -1}{\sin \theta } & =\lim _{\theta \rightarrow 0}\frac{\cos \theta -1}{\sin \theta } \times \frac{\theta }{\theta }\\
\uncover<2-6>{& =\lim _{\theta \rightarrow 0}\frac{\cos \theta -1}{\theta } \times \frac{\theta }{\sin \theta }\\}
\uncover<3-6>{& =\lim _{\theta \rightarrow 0}\frac{\cos \theta -1}{\theta } \times \lim _{\theta \rightarrow 0}\frac{\theta }{\sin \theta }\\}
\uncover<4-6>{& =\lim _{\theta \rightarrow 0}\frac{\cos \theta -1}{\theta } \times \frac{1}{\lim _{\theta \rightarrow 0}\frac{\sin \theta }{\theta }}\\}
\uncover<5-6>{& =0\times 1\\}
\uncover<6>{& =\mathbf{0}}
\end{align*}

\column{0.33\linewidth}

\onslide<3>{
\begin{tcolorbox}[colback=green!5,colframe=green!40!black,title=Theorem]
{\small $\lim_{\theta \rightarrow 0} \frac{\theta}{\sin \theta} = 1$ }
\end{tcolorbox}}

\onslide<4>{
\begin{tcolorbox}[colback=green!5,colframe=green!40!black,title=Theorem]
{\small $\lim_{\theta \rightarrow 0} \frac{\sin \theta}{\theta} = 1$ }
\end{tcolorbox}}
\end{columns}
\end{frame}

%%% Question 6 %%%
\begin{frame}[t]
\frametitle{Question 6}

\question{If $\displaystyle 2x\leq g( x) \leq x^{2} -x+2$ for all $\displaystyle x$, evaluate $\displaystyle \lim _{x\rightarrow 1} g(x)$.}

\vspace{1.5em}
$\lim _{x\rightarrow 1} 2x\leq \lim _{x\rightarrow 1} g(x)\leq \lim _{x\rightarrow 1} (x^{2} -x+2)$

\vspace{1.5em}
\uncover<2-3>{
Therefore, 
\begin{equation*}
2\leq \lim _{x\rightarrow 1} g(x)\leq 2
\end{equation*}}

\uncover<3>{
Hence,
\begin{equation*}
\lim _{x\rightarrow 1} g(x)=\mathbf{2}
\end{equation*}}

\end{frame}

%%% Question 7 %%% 
\begin{frame}[t]
\frametitle{Question 7}

\question{Solve $\displaystyle y'$ if $\displaystyle y=\sqrt{3x^{2} -2x+3}$.}

\vspace{1em}
Let $u=3x^{2} -2x+3$,\qquad therefore $\frac{du}{dx} =6x-2$.

\vspace{1em}
\uncover<2-7>{
Substitute $y=\sqrt{u} =u^{\frac{1}{2}}$, \qquad \ therefore $\frac{dy}{du} =\frac{1}{2} u^{-\frac{1}{2}}$.}

\vspace{0.5em}
\uncover<3-7>{
Putting back together,}
\begin{align*}
\uncover<3-7>{\frac{dy}{dx} & =\frac{dy}{du} \cdot \frac{du}{dx}\\}
\uncover<4-7>{& =\frac{1}{2} u^{-\frac{1}{2}} \cdot 6x-2\\}
\uncover<5-7>{& =\frac{1}{2} (3x^{2} -2x+3)^{-\frac{1}{2}} \cdot 6x-2\\}
\uncover<6-7>{& =\frac{6x-2}{2\sqrt{3x^{2} -2x+3}}}
\uncover<7>{=\boldsymbol{\frac{3x-1}{\sqrt{3x^{2} -2x+3}}}}
\end{align*}

\end{frame}

%%% Question 8 %%% 
\begin{frame}[t]
\frametitle{Question 8}

\scalebox{0.8}{\begin{minipage}{1.20\textwidth}

\question{Solve $\displaystyle y'$ if $\displaystyle y=5\sqrt[3]{x^{2} +\sqrt{x^{3}}}$.}

\vspace{1em}
$y=5\sqrt[3]{x^{2} +\sqrt{x^{3}}}$     can be rewrite as $y=5(x^{2} +x^{\frac{3}{2}} )^{\frac{1}{3}}$.

\vspace{1em}
\uncover<2-7>{
Let $u=x^{2} +x^{\frac{3}{2}}$, \qquad therefore $\frac{du}{dx} =2x+\frac{3}{2} x^{\frac{1}{2}}$.}

\vspace{1em}
\uncover<3-7>{
Substitute $y=5u^{\frac{1}{3}}$, \qquad therefore $\frac{dy}{du} =\frac{5}{3} u^{-\frac{2}{3}}$.}

\vspace{0.5em}
\uncover<4-7>{
Putting back together,}
\begin{align*}
\uncover<4-7>{\frac{dy}{dx} & =\frac{dy}{du} \cdot \frac{du}{dx}\\}
\uncover<5-7>{&=\frac{5}{3} u^{-\frac{2}{3}} \cdot \left( 2x+\frac{3}{2} x^{\frac{1}{2}}\right)}
\uncover<6-7>{=\boldsymbol{\frac{5}{3}\left( x^{2} +x^{\frac{3}{2}}\right)^{-\frac{2}{3}} \cdot \left( 2x+\frac{3}{2} x^{\frac{1}{2}}\right)}}
\end{align*}

\uncover<7>{
Alternatively, 
$\frac{5\left( 2x+\frac{3}{2}\sqrt{x}\right)}{3\left( x^{2} +\sqrt{x^{3}}\right)^{\frac{2}{3}}} =\frac{5\left(\frac{2x}{3} +\frac{1}{2}\sqrt{x}\right)}{\left( x^{2} +\sqrt{x^{3}}\right)^{\frac{2}{3}}} =\frac{5\left(\frac{2x}{3} +\frac{\sqrt{x^{3}}}{2x}\right)}{\left( x^{2} +\sqrt{x^{3}}\right)^{\frac{2}{3}}}$}

\end{minipage}}
\end{frame}

%%% Question 9 %%% 
\begin{frame}[t]
\frametitle{Question 9}

\question{Solve $\displaystyle y$ if $\displaystyle y=\ln\cos x^{2}$.}

\vspace{1em}
Let $u=\cos x^{2}$, \qquad therefore $\frac{du}{dx} =(-\sin x^{2} )(2x)$.

\vspace{1em}
\uncover<2-7>{
Substitute $y=\ln u$, \qquad therefore $\frac{dy}{du} =\frac{1}{u}$.}

\vspace{0.5em}
\uncover<3-7>{
Putting back together,}
\begin{align*}
\uncover<3-7>{\frac{dy}{dx} & =\frac{dy}{du} \cdot \frac{du}{dx}\\}
\uncover<4-7>{&=\frac{1}{u} \cdot (-\sin x^{2} )(2x)\\}
\uncover<5-7>{&=-\frac{2x\sin x^{2}}{\cos x^{2}}\\}
\uncover<6-7>{& =\frac{6x-2}{2\sqrt{3x^{2} -2x+3}}}
\uncover<7>{=\boldsymbol{-2x\tan x^{2}}}
\end{align*}

\end{frame}

%%% Question 10 %%% 
\begin{frame}[t]
\frametitle{Question 10}

\question{Differentiate $\displaystyle y=\log( 4+\cos x)$.}

\vspace{1em}
Note that $\frac{d}{dx} (\log_{b} (x))=\frac{1}{\ln (b)\cdot x} =\frac{1}{x} \cdot \log e$.

\vspace{1em}

\begin{align*}
\uncover<2-4>{\frac{d}{dx} (\log 4+\cos x) & =\frac{1}{4+\cos x} \cdot \log e\cdot \frac{d}{dx} (4+\cos x)\\}
\uncover<3-4>{& =\frac{1}{4+\cos x} \cdot \log e\cdot ( -\sin x)\\}
\uncover<4>{& =\boldsymbol{\frac{-(\log e)(\sin x)}{4+\cos x}}}
\end{align*}

\end{frame}

%%% Question 11 %%% 
\begin{frame}[t]
\frametitle{Question 11}

\question{Find $\displaystyle y'$ for $\displaystyle 10e^{2xy} =e^{15y} +e^{13x}$.}

\vspace{1em}
\begin{align*}
\uncover<2-6>{10e^{2xy} & =e^{15y} +e^{13x}\\}
\uncover<3-6>{10e^{2xy} (2x\cdot y'+2y) & =e^{15y} (15y')+e^{13x} (13)\\}
\uncover<4-6>{10e^{2xy} (2x\cdot y'+2y) & =15y'e^{15y} +13e^{13x}\\}
\uncover<5-6>{(20xe^{2xy} -15e^{15y} )y' & =13e^{13x} -20ye^{2xy}\\}
\uncover<6>{\boldsymbol{y'} & =\boldsymbol{\frac{13e^{13x} -20ye^{2xy}}{20xe^{2xy} -15e^{15y}}}}
\end{align*}

\end{frame}

%%% Question 12 %%% 
\begin{frame}[t]
\frametitle{Question 12}

\question{Find $\displaystyle f'( x)$ if $\displaystyle f( x) =2x(\arctan 5x)^{2} +6\tan(\cos 6x)$. }

\vspace{1em}
\begin{itemize}
    \item $\frac{d}{dx} (\tan^{-1} 5x)^{2} =2\tan^{-1} 5x\cdot \frac{1}{1+(5x)^{2}} \cdot 5=\frac{10\tan^{-1} 5x}{1+25x^{2}}$
    \uncover<2-4>{\item $\begin{aligned}[t]
    \frac{d}{dx} 2x(\tan^{-1} 5x)^{2} &=2x(\frac{10\tan^{-1} 5x}{1+25x^{2}} )+(\tan^{-1} 5x)^{2} \cdot 2\\
    &=\frac{20x\tan^{-1} 5x}{1+25x^{2}} +2(\tan^{-1} 5x)^{2}\end{aligned}$}
    \uncover<3-4>{\item $\begin{aligned}[t]
    \frac{d}{dx} 6\tan (\cos 6x)&=6\sec^{2} (\cos 6x)\frac{d}{dx}\cos 6x\\
    &=6\sec^{2} (\cos 6x)\cdot (-\sin 6x)\cdot 6\\
    &=-36(\sec^{2} (\cos 6x))\sin 6x\end{aligned}$}
\end{itemize}

\vspace{1em}
\uncover<4>{$\therefore f'(x)=\boldsymbol{\frac{20x\tan^{-1} 5x}{1+25x^{2}} +2(\tan^{-1} 5x)^{2} -36(\sec^{2} (\cos 6x))\sin 6x}$}


\end{frame}

%%% Question 13 %%% 
\begin{frame}[t]
\frametitle{Question 13}

\scalebox{0.8}{\begin{minipage}{1.20\textwidth}
\question{Solve $\displaystyle y'$ if $\displaystyle y=4x\sinh^{-1}\left(\frac{x}{6}\right) +\tanh^{-1}\left(\frac{x}{6}\right) +\tanh^{-1}(\cos 10x)$.}

\vspace{1em}
\begin{itemize}
    \item $\begin{aligned}[t]
    \frac{d}{dx} 4x\sinh^{-1} (\frac{x}{6} )&=4x\left[\frac{1}{\sqrt{1+(\frac{x}{6} )^{2}}} \cdot \frac{1}{6}\right] +\sinh^{-1} (\frac{x}{6} )\cdot 4\\
    &=\frac{\frac{2x}{3}}{\sqrt{1+\frac{x^{2}}{36}}} +4\sinh^{-1} (\frac{x}{6})\end{aligned}$
    \uncover<2-3>{\item $\begin{aligned}[t]
    \frac{d}{dx}\tanh^{-1} (\cos 10x)&=\frac{1}{1-(\cos 10x)^{2}} \cdot \frac{d}{dx}\cos 10x\\
    &=\frac{1}{1-\cos^{2} 10x} \cdot (-10\sin 10x)\\
    &=\frac{-10\sin 10x}{\sin^{2} 10x} \\
    &=-10\csc 10x\end{aligned}$}
\end{itemize}

\vspace{1em}
\uncover<3>{$\therefore f'(x)=\boldsymbol{\frac{\frac{2x}{3}}{\sqrt{1+\frac{x^{2}}{36}}} +4\sinh^{-1} (\frac{x}{6} )-10\csc 10x}$}
\end{minipage}}

\end{frame}

%%% Question 14 %%% 
\begin{frame}[t]
\frametitle{Question 14}

\question{Differentiate $\displaystyle y=\frac{1}{\sin^{-1} x}$.}

\vspace{1em}
\begin{tcolorbox}[colback=green!5,colframe=green!40!black,title=Theorem]
$\frac{d}{dx} (\sin^{-1} x)=\frac{1}{\sqrt{1-x^{2}}}$
\end{tcolorbox}

\vspace{1em}
\begin{align*}
\uncover<2-4>{y & =(\sin^{-1} x)^{-1}\\}
\uncover<3-4>{y' & =-(\sin^{-1} x)^{-2}\frac{d}{dx} (\sin^{-1} x)\\}
\uncover<4>{& =\boldsymbol{-\frac{1}{(\sin^{-1} x^{2} )\sqrt{1-x^{2}}}}}
\end{align*}

\end{frame}

%%% Question 15 %%% 
\begin{frame}[t]
\frametitle{Question 15}

\question{Differentiate $\displaystyle y=\left( x^{3} -1\right)^{100}$.}

\vspace{1em}
\begin{align*}
y' & =100(x^{3} -1)^{99}\frac{d}{dx} (x^{3} -1)\\
\uncover<2-3>{& =100(x^{3} -1)^{99} \cdot 3x^{2}\\}
\uncover<3>{& =\boldsymbol{300x^{2} (x^{3} -1)^{99}}}
\end{align*}

\end{frame}

\end{document} 